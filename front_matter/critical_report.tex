\documentclass[abbrwidth=6em,tocstyle=ref-genre,shorttitlesize=50]{ees}

\begin{document}

\eesTitlePage

\eesCriticalReport{
  –   & –     & org  & Bass figures appear in the following movements of \B1 (bars in parentheses): 1.1, 1.3, 1.4 (14 to 16, 26), 1.5 (9), 1.6 (2 to 7), 2.1, 2.3 (4, 5, 12 to 19, 25, 29, 30, 51 to 60), 2.4 (4), 2.6, 3.1, 3.2 (17 to 23, 26), 3.3 (1, 6, 8, 29), 3.4 (12 to 15, 20), 4.1, 4.3 (16 to 18, 21 to 23, 47 to 51, 61 to 66), 4.5, 4.6, 5.1, 5.3, 5.4 (17 to 19), 5.5 (1 to 49), 5.6, 6.1, 6.3, 6.4 (8, 9, 11, 12, 21, 24 to 34), 6.5, and 6.6. The remaining bass figures were added by the editor. \\
  1.1 & 7     & A    & 2nd \halfNote\ in \B1: g′2 \\
  1.5 & 21    & vl 2 & bar in \B1 unison with vl 1 \\
  2.3 & 8     & ob 1 & last \sixteenthNote\ in \B1: c″32–d″32 \\
  2.5 & 20    & vl 1 & 2nd \quarterNote\ in \B1: \sharp c″8–d″16–e″16 \\
      & 71    & vl 1 & 2nd \quarterNote\ in \B1: b′8–c″16–d″16 \\
  2.6 & 4     & T    & 5th \halfNote\ in \B1: c′4–a′4 \\
  3.3 & 22    & vl 1 & 7th \eighthNote\ in \B1: a′8 \\
  3.5 & 16    & A    & 6th \eighthNote\ in \B1: f′8 \\
      & 26    & vlc  & 4th \quarterNote\ missing in \B1 \\
  3.6 & 1     & T    & 4th \halfNote\ in \B1: \flat b2 \\
      & 2     & T    & 3rd \wholeNote\ in \B1: d′2.–d′4 \\
  4.3 & 34–37 & vl   & vl 1 and 2 swapped in \B1 \\
      & 51    & B    & 2nd/3rd \quarterNote\ in \B1: d′8–d′8–\sharp c′8–b8 \\
  5.3 & 33    & org  & last \sixteenthNote\ in \B1: d′16 \\
      & 41    & ob 1 & last \eighthNote\ in \B1: a″16–b″16 \\
      & 45    & org  & 2nd \quarterNote\ in \B1: d16–g16–b16–d′16 \\
  5.5 & 64    & fag  & bar in \B1: c′8–g4 \\
      & 66    & fag  & 9th note in \B1: c′32 \\
  5.6 & 1–2   & –    & These bars are missing in the \B1 and were reconstructed based upon movement 1.22 in Stölzel’s passion \textit{Der gute Hirte}. \\
  6.3 & 20    & vl 2 & 5th \quarterNote\ in \B1: f′16–f′16–f′16–f′16 \\
      & 23    & vl 1 & 1st \quarterNote\ in \B1: e″16–\sharp d″16–\sharp d″16–\sharp d″16 \\
  6.5 & –     & ob   & According to \B1, two flutes may be used instead of ob da caccia (“Grand=Oboi, ſtatt denen 2 Flöten, welche a part geſchrieben ſind”). \\
      & –     & A, T & Original lyrics in \B1 have been crossed out. Here, they are reproduced as second stanza. \\
      & 57    & org  & bar in \B1: G2. \\
}

\eesToc{
\part{erstebetrachtung}

\begin{movement}{jesudeine}
  \voice[Coro]
  Jeſu, deine Paßion\\
  will ich itzt bedencken,\\
  wolleſt mir vom Himmelsthron\\
  Geiſt und Andacht ſchencken.\\
  In dem Bild itzund erſchein,\\
  Jeſu, meinem Hertzen,\\
  wie du, unſer Heil zu ſeyn,\\
  litteſt alle Schmertzen.
\end{movement}

\begin{movement}{sothustdu}
  \voice[Tenore]
  So thuſt du dann,\\
  nach dem geſprochnen Lobgeſang,\\
  mein Heyland, über Kidrons Fluth\\
  den erſten Leidens Gang.\\
  Dir folgen deine Jünger zwar,\\
  und faßen den feſten Schluß,\\
  mit tapfern Muth dich nicht in der Gefahr,\\
  ja nicht im Tode zu verlaßen.\\
  Doch du ſagſt es vorher,\\
  daß er von ſchlechter Dauer ſey,\\
  und giebſt das zweyte Hahngeſchrey\\
  zu einem Zeichen,\\
  daß Petrus ſelbſt zurücke werde weichen.
\end{movement}

\begin{movement}{wiehurtig}
  \voice[Tenore]
  Wie hurtig iſt man im Verheißen,\\
  wie leicht gelobt man alles an.\\
  Doch wenn mans ſoll im Werck erweiſen,\\
  ſo bleibt es leider ungethan.
\end{movement}

\begin{movement}{dutrittst}
  \voice[Alto]
  Du trittſt, mein Jeſu, nun in Hof Gethſemane\\
  und uns demſelben in den Garten,\\
  wo Todes Angſt und Seelen Weh,\\
  wo Band und Stricke deiner warten.

  \voice[Soprano]
  Du fällſt betrübt auf Knie und Angeſicht,\\
  fängſt an zu zittren und zu zagen\\
  vor der entſetzlich großen Laſt,\\
  die Du auf deinen Rücken haſt\\
  und nun, o Gottes Lamm, ſollſt tragen.

  \voice[Alto]
  Dreymal ſteigt dein Gebeth\\
  zum Vater in die Höhe,\\
  daß dieſer Kelch doch von dir gehe,\\
  doch indes wohlruht auch dein Wille\\
  im Willen deines Vaters ſtille.

  \voice[Soprano]
  Itzt ſtärcket dich ein Himmelsbothe.\\
  Drauf ringſt du mit dem Tode,\\
  ſodaß dein Schweiß, o Gottes Held,\\
  wie Tropfen Bluts zur Erde fällt.

\voice[both]
  Iſts möglich, daß nun diesen,\\
  die ihren Beyſtand dir verhießen,\\
  ein tiefer Schlaf die Augen will verſchließen.
\end{movement}

\begin{movement}{wachtundbetet}
  \voice[Soprano, Alto]
  Wacht und bethet! ruffeſt du\\
  den betrübten Jüngern zu\\
  und zugleich auch allen Chriſten.\\
  Wer mit dir den Leidens Plan\\
  als ein treuer Unterthan,\\
  liebſter Jeſu, will betreten,\\
  muß mit Wachen und mit Bethen\\
  Augen, Hertz und Seele rüſten.
\end{movement}

\begin{movement}{wachetauf}
  \voice[Chor]
  Wachet auf, ihr faulen Chriſten,\\
  bedencket, daß euch Gottes Gnad\\
  vom tiefen Schlaf der Sünden Lüſten\\
  zu leben auferwecket hat.\\
  Verlaßet doch die finſtre Gruft\\
  und höret, wenn euch Jeſus ruft:\\
  Wachet!
\end{movement}

\clearpage
\part{zweytebetrachtung}

\begin{movement}{judasder}
  \voice[Coro]
  Judas, der Verräther, küßt\\
  Jeſum, ach! er wird gefangen.\\
  In den Banden Chriſtus büßt\\
  was der [Sünden] Mensch begangen.\\
  Zu des Himmels Freyheits Thür,\\
  trauter Jeſu, geh uns für.
\end{movement}

\begin{movement}{waskoemmestu}
  \voice[Alto]
  Was kömmeſtu, verruchte Schaar\\
  mit Fackeln, langen Schwerd und Stangen?\\
  Mein Heyland läßt ſich willig fangen.\\
  Es braucht es nicht, untreuer Böſewicht,\\
  daß dein verrätheriſcher Kuß\\
  das Zeichen geben muß,\\
  wer Jeſus ſey:\\
  Er ſagt es ſelber ohne Scheu\\
  und reichet ſeine Hand den Banden dar.\\
  Laß ſtecken nur das Schwerd in ſeiner Scheiden,\\
  jetzt iſt die Stunde da, daß Jeſus leide;\\
  ſonſt würde ja ein Himmels Heer\\
  von Engel Legionen\\
  der böſen Rotte nicht verſchonen.
\end{movement}

\begin{movement}{liebstehand}
  \voice[Alto]
  Liebſte Hand, ich küße dich,\\
  denn du läßeſt auch vor mich\\
  dich mit Banden hart belegen.\\
  Ich gehörte ewiglich\\
  meiner Mißethaten wegen\\
  in der Hölle Folterhauß,\\
  doch du zieheſt mich heraus,\\
  liebſte Hand!
\end{movement}

\begin{movement}{undwieihr}
  \voice[Tenore]
  Und wie? Ihr Jünger, fliehet ihr?\\
  Iſt euer Muth nun ganz und gar verſchwunden?\\
  Bedenckt, wozu ihr euch verbunden!

  \voice[Basſo]
  Mein Heyland, ja ſo geht es dir,\\
  dein innerliche Seelen Plagen,\\
  ſo dich bis an den Todt betrübt,\\
  dein Trauren, Zittern und dein Zagen,\\
  den blutgen Todes Schweiß,\\
  der tropffenweiß von deinen Angeſichte rann,\\
  ſteht man ja wohl noch ſchläfrig an.\\
  Doch jetzt, da dich kaum äußerlich\\
  der Feinde Schaar umgiebt,\\
  und Strick und Bande dich umfaßen,\\
  flieht man und weiß ſich nicht vor Furcht zu laßen.
\end{movement}

\begin{movement}{achihrjuenger}
  \voice[Tenore, Basso]
  Ach, ihr Jünger, denckt zurücke,\\
  was ſind dieſe Band und Stricke\\
  gegen Jeſu Seelen Schmertz.\\
  Jeſum gantz alleine traffe\\
  aller Menſchen Sündt und Strafe,\\
  Höllenbande, Todesſtricke\\
  folterten ſein blutend Hertz.
\end{movement}

\begin{movement}{herrlass}
  \voice[Coro]
  Herr, laß dein bitter Leyden\\
  mich reitzen für und für,\\
  mit allem Ernſt zu meyden\\
  die ſündliche Begier:\\
  daß mir nie komme aus dem Sinn\\
  wie viel es dich gekoſtet,\\
  daß ich erlöſet bin.
\end{movement}

\part{drittebetrachtung}

\begin{movement}{jesuderdu}
  \voice[Coro]
  Jeſu, der du wollen büßen\\
  vor die Sünden aller Welt\\
  durch dein theures Blutvergißen,\\
  der du dich haſt dargeſtellt\\
  als ein Opffer vor die Sünder,\\
  die verdamten Adamskinder,\\
  ach! laß deine Todes Pein\\
  nicht an mir verlohren ſeyn.
\end{movement}

\begin{movement}{schauhannas}
  \voice[Tenore]
  Schau, Hannas, den,\\
  der für des Volckes Mißethaten,\\
  wie Kaiphas aus Boßheit zwar gerathen,\\
  ja für das Heyl der gantzen Welt\\
  ſich jetzt zum Gaffer ſtelt.\\
  Was fragſtu, Kaiphas, nach ſeiner Lehre,\\
  als ob ſie dir verdächtig wäre?\\
  Hör, was der Heyland ſpricht:

  \voice[Basso]
  Hat ſie der Jüden Schule nicht,\\
  hat ſie der Tempel nicht gehört,\\
  hab ich nicht frey,\\
  nicht öffentlich gelehrt?\\
  Befrage dieſe, ſo es wißen\\
  und von denſelben zeigen müßen.

  \voice[Tenore]
  Und welche Raſerey\\
  bewegt doch deine Hand,\\
  verdammter Höllenbrand,\\
  durch einen Backenſtreich\\
  den Heyland zu bedeuten?\\
  Er rede allzu frey\\
  und alßo unbeſcheiden?\\
  Ihr Prieſter, was bemüht ihr euch\\
  um falſche Zeugen,\\
  erkennt ihr ihre Lügen nicht\\
  an Jeſu Stilleſchweigen?\\
  Doch da er ohne Scheu\\
  auf eure Fragen ſpricht,\\
  daß er ein Sohn des großen Gottes ſey:

  \voice[both]
  Was ſchlägt man ihn,\\
  as ſpeyt man ihm\\
  ins heilge Angeſicht!
\end{movement}

\begin{movement}{wehedir}
  \voice[Tenore, Basso]
  Wehe dir verruchten Volcke,\\
  wenn er auf dem Thron der Wolcke\\
  einſt als ſtrenger Richter ſitzt.\\
  Welcher Hügel wird dich decken,\\
  welcher Berg wird dich verſtecken,\\
  wenn ſein ſcharffes Rachſchwerd blitzt?
\end{movement}

\begin{movement}{undpetre}
  \voice[Alto]
  Und Petre, du erkühneſt dich,\\
  den Heyland, aber nur von weiten,\\
  biß in Pallaſt des hohen Prieſters zu begleiten.\\
  Ach! daß dir hier der Muth entwich!\\
  Ach! daß du nun kein Hertze haſt,\\
  wenn Knecht und Mägde dich befragen,\\
  die reine Wahrheit anzuſagen,\\
  daß du ein Jünger Jeſu ſeyſt!\\
  Trifft deines Meiſters Wort nun ein,\\
  daß zwar dein Geiſt ſo willig werde ſeyn,\\
  als ſchwach du nach dem Fleiſche biſt.\\
  Merck auf! jetzt kreht zum zweytenmal der Hahn,\\
  der deines Falls beredter Zeuge iſt.\\
  Schau! Jeſus wendet ſich,\\
  ſeyn Blick erinnert deßen dich,\\
  was du nun dreymal ſchon gethan.
\end{movement}

\begin{movement}{japetre}
  \voice[Alto]
  Ja, Petre, geh hinaus und weine,\\
  ja, weine, weine bitterlich.\\
  Denck zwar an deinen Fall zurücke,\\
  doch dencke auch an Jeſu Blick,\\
  in dieſes Blickes holden Scheine\\
  zeigt deine Gnadenſonne ſich.
\end{movement}

\begin{movement}{ichbitt}
  \voice[Coro]
  Ich bitt, ich ruf, ich weine,\\
  Herr Jeſu, wende dich,\\
  wie Petro mir erſcheine,\\
  und bring zur Ruhe mich.\\
  Ich traue deinem Sterben,\\
  nimm meiner Seel dich an,\\
  ach laß die nicht verderben,\\
  für die du gnug gethan.
\end{movement}

\part{viertebetrachtung}

\begin{movement}{jesusfuer}
  \voice[Coro]
  Jeſus für Pilato ſteht,\\
  falſch beklaget er da leidet,\\
  drauf hin zu Herodes geht,\\
  weiß die Unſchuld ihn bekleidet.\\
  Ach! in Unſchuld dort und hier,\\
  trauter Jeſu, geh uns für.
\end{movement}

\begin{movement}{kaumwirdmein}
  \voice[Soprano]
  Kaum wird mein Jeſus frühe\\
  vors weltliche Gericht geführet,\\
  daß man an ihm den Todesſpruch vollziehe;\\
  als Judas Höllenangſt verſpürt,\\
  daß er unſchuldig Blut verrathen.

  \voice[Basso]
  Er ſiehet ſeine Mißethaten,\\
  die er um ſchnödes Geld gethan,\\
  wie Kain ſeine Mordthat an,\\
  und bringt den Lohn der Ungerechtigkeit zurücke,\\
  hebt ſich verzweifflungsvoll davon\\
  und endet ſeine Reu an einem Stricke.
\end{movement}

\begin{movement}{duverschmitzter}
  \voice[Soprano, Basso]
  Du verſchmitzter Schlangengriff,\\
  o wie manches Glaubensſchiff\\
  ſenckeſt du in tiefſten Abgrund\\
  der Verzweiffelung hinein.\\
  Erſt machſt du die Sünden klein,\\
  daß wir keine Furcht empfünden,\\
  aber aus vollbrachten Sünden\\
  als die Gnade ſollen ſeyn.
\end{movement}

\begin{movement}{esscheuen}
  \voice[Tenore]
  Es ſcheuen zwar die Jüden die Gefahr,\\
  ins Richthaus einzugehn,\\
  daß ſie nicht unrein würden;\\
  doch Jeſum zu verklagen,\\
  ihm tauſend Schulden aufzubürden,\\
  ihm vor Gerichte nachzuſagen,\\
  daß er ein Übelthäter ſey,\\
  das thun ſie ohne Scheu.\\
  So offt als nur Pilatus ſpricht:\\
  Ich finde keine Schuld an dieſem Menſchen nicht,\\
  ſo offt hallt ihr Geſchrey umhan,\\
  er habe diß und das gethan.\\
  Herodes ſelbſt beweiſt\\
  mit einem weißen Kleide,\\
  womit er ihn bekleiden heißt,\\
  daß Jeſus unverſchuldet leide.\\
  Indennoch wird ein Barrabas,\\
  der Mord und Aufruhrs wegen\\
  bereits in Ketten ſaß,\\
  vom Richter loß geſprochen,\\
  und über Jeſum wird hingegen,\\
  nachdem man ihn verſpeyt,\\
  gegeißelt und verhöhnt,\\
  nachdem man ihn mit Dornen gar gekrönt,\\
  der Todesſtab gebrochen.
\end{movement}

\begin{movement}{unbeflecktes}
  \voice[Tenore]
  Unbeflecktes Gotteslamm,\\
  nicht alleine Barrabam,\\
  den verruchten Mißethäter,\\
  macheſt du vom Tode frey.\\
  Denn du ſtirbſt am Creutzes Stamm,\\
  daß der Aufruhr, den ich dort\\
  in dem Odem angefangen,\\
  und der allgemeine Mord,\\
  welchen ich mit ihn begangen,\\
  uns durch dich, du Schlangentreter,\\
  völlig ausgeſöhnet ſey.
\end{movement}

\begin{movement}{duspringst}
  \voice[Coro]
  Du ſpringst ins Todes Rachen,\\
  mich frey und loß zu machen\\
  von ſolchem Ungeheur:\\
  Mein Sterben nimmſt du abe,\\
  vergräbſt es in dem Grabe.\\
  O! unerhöhrtes Liebes Feur.
\end{movement}

\part{fuenftebetrachtung}

\begin{movement}{hinwegihr}
  \voice[Coro]
  Hinweg, ihr irrdſchen Hindernißen\\
  mit eurem trüglich falſchen Schein,\\
  mein Hertz ſoll anderſt nichtes wißen,\\
  als meinen Jeſum gantz allein,\\
  wie er, von meiner Schuld bedränget,\\
  erbärmlich an dem Creutze hänget.
\end{movement}

\begin{movement}{sotraegst}
  \voice[Alto]
  So trägſt du denn des Creutzes ſchwere Bürde,\\
  mein Jeſus, ſelbſt nach Golgatha.\\
  Ach, daß ich williglich\\
  ein Simon von Cyrene würde\\
  und nähm dein Joch auf mich.\\
  Ach, folgt ich dir in deiner Schmach\\
  biß an die Schädelſtätte nach,\\
  wo Eſig, Myrrhen, Gall und Wein\\
  dein erſter Labtrunck ſolten ſeyn;\\
  wo man nach unerhörten Qualen\\
  dich an den Creutzes Pfahl erhöht,\\
  der zwiſchen zween Creutzes Pfählen\\
  der ärgſten Übelthäter ſteht;\\
  wo eines Heyden Schrifft\\
  zu deinen Haupte weißt,\\
  daß du der Jüden König ſeyſt;\\
  wo man das Loß um deine Kleider wirft,\\
  daß man ſie nicht zertheilen dürfft;\\
  wo du von deinen Creutz\\
  ein traurig Paar erblickeſt\\
  und ſolches beyderſeits\\
  mit Rath und Troſt erquickeſt,\\
  wenn du den Jünger, den du liebſt,\\
  der Mutter nun zum Sohn,\\
  ſie ihm zur Mutter giebſt.
\end{movement}

\begin{movement}{verloeschtdenn}
  \voice[Alto]
  Verlöſcht denn deine Liebe nicht,\\
  da um dein bloßes Angeſicht\\
  bereits die Todes Schatten ſchweben?\\
  Nein, du willſt noch vor deinem Todt\\
  der Liebe, holdes Abendroth,\\
  o Gnadenſonne, von dir geben.
\end{movement}

\begin{movement}{womiterquickt}
  \voice[Soprano]
  Womit erquickt man Jeſum nun,\\
  nachdem die ärgſten Läſterungen\\
  von tauſend Läſterzungen\\
  wie Pfeile durch ſein ſterbend Hertz gedrungen?

  \voice[Tenore]
  Soll es der ſcharffe Eſig thun,\\
  den ihm ein Kriegsknecht bringt?\\
  Nein, grauſame, nein, nein!

  \voice[Soprano]
  Iſts möglich, daß ſein Hertze,\\
  das mit dem Tode ringt,\\
  noch eh es völlig bricht\\
  etwas erfreuen kann,\\
  ſo wird es dieſes ſeyn,\\
  daß unter Glauben, Reu und Schmertz\\
  der eine Übelthäter ſpricht:\\
  Herr, kömmeſt du in deinem Reiche an,\\
  ach, ſo gedencke mein.
\end{movement}

\begin{movement}{owietroestlich}
  \voice[Soprano, Tenore]
  O wie tröſtlich, o wie ſüße\\
  klingt die Antwort Jeſu drauf,\\
  welche ihm im Paradieſe\\
  nach geſchloßnen Lebenslauf\\
  einen Aufenthalt verhieße.
\end{movement}

\clearpage
\begin{movement}{sowahrhaftig}
  \voice[Coro]
  So wahrhaftig als ich lebe,\\
  will ich keines Menſchen Todt,\\
  ſondern, daß er ſich ergebe\\
  an mir aus dem Sündenkoth.\\
  Gottes Freud ist, wenn auf Erd\\
  ein Verirrter wiederkehrt,\\
  will nicht, daß aus ſeiner Heerde\\
  das Geringſt entzogen werde.
\end{movement}

\part{sechstebetrachtung}

\begin{movement}{kommtihr}
  \voice[Coro]
  Kommt, ihr Geschöpffe, kommt herbey,\\
  und machet bald ein Klaggeſchrey,\\
  das grauſam ſey zur ſelben Frist,\\
  da Gott am Creutz verſchieden iſt.
\end{movement}

\begin{movement}{verbirg}
  \voice[Alto]
  Verbirg, o Sonne, nur das Licht\\
  von deinem Angeſicht\\
  in tieffe Finſterniße,\\
  dein froher Schein\\
  will kein betrübter Zeuge ſeyn,\\
  daß itzt mein Jeſus klagen müße,\\
  er ſey, wer kan das Elend faßen,\\
  von Gott verlaßen.

  \voice[Tenore]
  Ach! flößt ihr ihm in ſolcher ſchwerer Pein\\
  nach Yſopen nur Eßig ein,\\
  als wäret ihr bedacht,\\
  ihn noch vor ſeinem Ende,\\
  ihr Mörder, zu erquicken.

  \voice[Soprano]
  Er nimmt es dennoch an und rufft:\\
  Es iſt vollbracht,\\
  befiehlt den Geiſt in ſeines Vaters Hände,\\
  fängt an, die Augen zuzudrücken,\\
  neigt das erblaßte Haupt,\\
  und gibt dem Leben gute Nacht.
\end{movement}

\begin{movement}{oeingrosser}
  \voice[Coro]
  O, ein großer Todesfall,\\
  Jeſus höret auf zu leben.\\
  Zittre, ganzer Erdenball,\\
  berſtet auf, ihr Felſenklüfte,\\
  öffnet euch, ihr Todtengrüffte,\\
  unter einen Donnerknall\\
  weit und breit, ja überall,\\
  Todesboten abzugeben.
\end{movement}

\begin{movement}{wasduencket}
  \voice[Soprano]
  Was düncket euch,\\
  die ihr beym Creutze Jeſu steht,\\
  und dieſe Trauerzeichen ſeht?

  \voice[Alto]
  Könnt ihr uns dergeſtallt\\
  der traurigen Natur nunmehro leſen?\\
  Mein Jeſus ſey ein frommer Menſch\\
  und Gottes Sohn geweſen.

  \voice[Tenore]
  Ihr mörderiſchen Jüden erſucht Pilatum für,\\
  daß er die Bein ihm laße brechen.\\
  Die Schrifft will euch hierinnen widerſprechen:

  \voice[Basſo]
  Mein Jeſus iſt bereits verſchieden,\\
  an dem erwürgten Oſterlamme\\
  bricht man die Beine nicht.

  \voice[Soprano]
  Ja weil die Schrifft noch weiter ſpricht:\\
  Sie werden ſehn, in welchen ſie geſtochen haben,\\
  ſo muß ein Speer durch Jeſu Seite gehn\\
  und hier den Lebensbrunnen graben,\\
  aus welchen Blut und Waßer quillt,\\
  ſo mir den Durſt der Seele ſtillt.

  \voice[Alto]
  Der Abend kömmet nun.\\
  Wer nimmt das Lamm vom Creutzes Stamme?

  \voice[Tenore]
  Ein Joſeph will es thun,\\
  ein Nicodemus ſteht ihm bey.

  \voice[Alto, Tenore]
  Die nehmen Jeſu Leichnam ab,\\
  umbinden ihn mit Specerey\\
  und wickeln ihn in reinen Leinwand ein,\\
  verwahren ihn im Grab\\
  mit einem großen Stein.
\end{movement}

\begin{movement}{gottversoehner}
  \voice[Alto, Tenore]
  Gott Verſöhner, ſanft im Schlummer\\
  ruhſt du nach vollbrachten Leiden\\
  in der ſtillen Todesnacht.\\
  Nun kann ich befreit von Kummer\\
  glaubensvoll und ſanfft verſcheiden.\\
  Durch dich kann ich Rettung hoffen,\\
  du zeigſt mir den Himmel offen.\\
  Tod, wo iſt nun deine Macht!

  [original lyrics, crossed out:]\\
  Weyh durch deine Grabesſtätte,\\
  Jeſus, nun das Sterbebette\\
  einſt zum ſanfften Lager mir.\\
  Schlägt in dieſem finſtren Thale\\
  mir das Hertz zum letzten Mahle\\
  und du ruffeſt mir entgegen,\\
  daß du auch im Grab gelegen,\\
  o wie ruhig werd ich ſeyn!
\end{movement}

\begin{movement}{javerlass}
  \voice[Coro]
  Ja, verlaß die finſtre Hölle\\
  deiner ſchwartzen Todten Gruft,\\
  komm und ruh in meiner Seele,\\
  die gantz ſehnlich nach dir ruft.\\
  Komm und ſey mir ſtets im Sinne,\\
  bis ich dich recht lieb gewinne.
\end{movement}
}

\eesScore

\end{document}
